\documentclass{article}
\usepackage{rubylatex}

\begin{document}

Hello! This is an example of using $\RbTeX$! The goal of $\RbTeX$ is to provide dynamic functionality during
compilation of \LaTeX\ documents. Let's get started!\\

\begin{rbtex}
    #Tex.print "Hi! I'm being printed from a Tex.print line inside Ruby."
    #Tex.print "I really like to say hello, so I'm going to say it a lot!"

    #Tex.print(Tex.n << Tex.n)

    #10.times do
    #    Tex.print "Hello, friend!"
    #end

    #Tex.print "#{Tex.n}#{Tex.n}"

    #Tex.print "I can also easily typeset math. Unfortunately, you will have to escape the backslash character, but that's not too bad."

    #Tex.print("My favorite equation is " << Tex.imath("f(x) = x^{2}" << "!"))
    #Tex.print("This isn't really the case, but it is a nice equation.")
    #Tex.print(Tex.logo << " supports all of the paradigms of Ruby. For example, I can create classes and use them later.")

    #Tex.print "#{Tex.n}#{Tex.n}"

    class MyClass

        def initialize
            Tex.print("Hi! I'm a class being initialized.")
        end

    end

    mcl = MyClass.new
\end{rbtex}
\end{document}
