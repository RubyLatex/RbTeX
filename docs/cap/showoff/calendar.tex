\documentclass{hw}
\usepackage{rubylatex}

% \newcolumntype{C}[1]{>{\centering}m{#1}}
\newcolumntype{L}[1]{>{\raggedright\let\newline\\\arraybackslash\hspace{0pt}}m{#1}}
\newcolumntype{C}[1]{>{\centering\let\newline\\\arraybackslash\hspace{0pt}}m{#1}}
\newcolumntype{R}[1]{>{\raggedleft\let\newline\\\arraybackslash\hspace{0pt}}m{#1}}

\begin{document}

\begin{rbtex}
overallTime = Time.now
monthName = overallTime.strftime("%B")
Tex.print("\\section*{#{monthName}, #{overallTime.year}}")

def dayOfWeek(id)
    return case (id % 7)
        when 0
            "Monday"
        when 1
            "Tuesday"
        when 2
            "Wednesday"
        when 3
            "Thursday"
        when 4
            "Friday"
        when 5
            "Saturday"
        else
            "Sunday"
        end
end
\end{rbtex}
\begin{tabular}{|p{2cm}|p{2cm}|p{2cm}|p{2cm}|p{2cm}|p{2cm}|p{2cm}|}
\hline
\begin{rbtex}
i = 1
nlnstr = "\\\\ \\hline"
7.times {
    Tex.print("#{dayOfWeek(i+3)}#{i % 7 == 0 ? nlnstr : '&'}")
    i = i + 1
}
i = 1
days = 31
nlnstr = "\\\\[2cm] \\hline"
5.times {
    7.times {
        Tex.print("#{i <= days ? i : ''}#{i % 7 == 0 ? nlnstr : '&'} ")
        i = i + 1
    }
}
\end{rbtex}
\end{tabular}

\end{document}
