%% Logbook should be in the format
%% Task | Hours | Notes | Done/Not Done
%%
%% Project must be under git revision

\documentclass{article}

\usepackage{amsmath}
\usepackage{listings}
\usepackage{graphicx}

\newcommand{\inlinecode}[1]{\texttt{#1}}
\newcommand{\luatex}{\inlinecode{lualatex}\ }
\newcommand{\findent}{\leavevmode{\parindent=1.3em\indent}}

\def\RbTeX{{\rm\kern-.125emR\!{\scalebox{0.7}{\lower-.5ex\hbox{B}}}\!T\kern-.1667em\lower.5ex\hbox
{E}\kern-.125emX}}

\topmargin=-1in
\textheight=9.2in
\textwidth=168mm
\oddsidemargin=-0.2in
\evensidemargin=-0.2in

\pagenumbering{gobble}
\title{CPSC 498 Proposal: \RbTeX}
\author{Steven Rosendahl}
\date{}
\begin{document}
\maketitle
\section{Abstract}

\findent Modern \LaTeX\ distributions include a tool called \luatex that allows users to
dynamically produce content via use of Lua code. Unfortunately, the Lua standard libraries do not
have as much functionality as other popular scripting languages, such as Ruby. The goal of this
project is to incorporate Ruby into \LaTeX\ in a manner similar to \luatex, but with the power and
simplicity of Ruby over Lua.\\

One of the main benefits of having the ability to dynamically produce \LaTeX\ code is that
difficult mathematical problems can be solved using a high level language. Ruby's syntax is
simplistic enough to make up for the fact that the \TeX\ syntax is verbose; a gem will be provided
to perform some mathematical operations and present the result in \LaTeX.

\section{Motivation and Description}
\subsection{\LaTeX and Ruby}

\findent The current \luatex specification allows users to use several environments for writing
and running Lua scripts. In addition, \luatex provides a built in library called \inlinecode{tex}
that allows output to be printed straight to the \LaTeX\ document. The library, called \RbTeX,
will provide similar functionality through a gem called \inlinecode{rbtex}. In addition, the
entire Ruby standard library will be available for use; \RbTeX\ documents that need to interact
directly with the system will most likely need to be compiled using the
\inlinecode{--shell-escape} flag.\\

To use the library, users will need to have a Ruby version in the path. The code will be
pre-processed, and inserted directly into the \TeX\ code before \inlinecode{pdflatex} is called on
the document. In addition, users will be provided with several ways in which to interact with Ruby
from the \TeX\ environment:

\begin{enumerate}
\item \inlinecode{rbtex\{\}}: This command will provide a way to write multiple lines of Ruby code
inside the \LaTeX\ document. Any functions defined in this section will be globally defined, so
they can be called in the \inlinecode{inrbtex\{\}} environment and in other \inlinecode{rbtex\{\}}
environments. This code will be pulled verbatim from the \LaTeX\ file during the pre-processing
stage.
\item \inlinecode{frbtex\{\}}: It may sometimes be convenient for a user to write an external Ruby
file and call it from the \LaTeX\ file, rather than writing the code. The \inlinecode{frbtex\{\}}
macro will allow an external script to be loaded into the document. The pre-processor will copy
the provided Ruby file into the \LaTeX\ document, and will assume that all modules, classes,
functions, and variables are globally defined.
\end{enumerate}

The library will come with a program called \inlinecode{rbtex} that complies the provided \LaTeX\
document, much like the \inlinecode{luatex} command.\\

The program will work in four steps. It will first pre-process the \TeX\ file, scanning for the
appropriate environments. The ruby code will be ripped out and stored in an \inlinecode{.aux}
file. The next step will order the code in the \inlinecode{.aux} file, and attempt to produce a
\inlinecode{.rb} file from the provided \LaTeX\ document. In the third step, the code in the
\inlinecode{.rb} file will be run using the \inlinecode{ruby} command specified in the user's
\inlinecode{\$PATH} variable. Finally, the post-processor will capture any output specified by the
\inlinecode{tex} module in the \inlinecode{rbtex} gem, and place it into the \LaTeX\ document.
From there, \inlinecode{pdflatex} will take over.\\

The pre-processor and post-processor will be written in C++ to allow for quick speed when parsing
out the Ruby code. Shell script (UNIX) and Batch files (Win) will be provided for calling the
program. Standard \inlinecode{pdflatex} flags will be available (they will be simply passed to the
\inlinecode{pdflatex} command at the appropriate time). The final version of the program will be
accessible through the shell command \inlinecode{rbtex tex\_file.tex} and the Windows equivalent.

\subsection{Ruby and Math}
Ruby includes several mathematical utilities that are sufficient to solve simple math problems.
However, Ruby lacks the power to solve higher level problems, such as Ordinary Differential
Equations and even Partial Differential Equations. This package will include a gem called
\inlinecode{rbtexm} that will have several features geared towards mathematics:
\begin{enumerate}
\item ODE Solver: The gem will include a module that can solve a wide variety of first and second
order ODE's. If it runs into an ODE that it cannot solve, it will gracefully exit.
\item PDE Solver: This module will allow for simple first and second order PDE's to be solved. It
will have functionality to solve only homogeneous PDE's.
\item Integration: This module will allow for integration of functions, when possible.
\item Differentiation: A simple module to differentiate functions.
\item Logic: This module will be able to solve binary logic problems. It will return a truth table
that has all possible outcomes.
\end{enumerate}

Time permitting, more mathematical modules will be added. All the models will provide a way to
output \LaTeX\ code for use in the document; however, the math module should be able to be used
independent of \LaTeX. The methods available will take in strings of \LaTeX\ code. The gem will
either contain a small \LaTeX\ parser, or will include a third party one.

\section{Tentative Schedule}
\begin{center}
\begin{tabular}{| c | l |}
\hline
January 19 & Proposal and initial git setup\\
\hline
January 26 & Building C++ preprocessor\\
\hline
February 2 & Building C++ postprocessor\\
\hline
February 9 & Building \inlinecode{rbtex} gem\\
\hline
February 16 & Begin writing \inlinecode{rbtexm} gem\\
\hline
February 23 & Build small \LaTeX\ parser\\
\hline
March 1 & Build calculus related utilities\\
\hline
March 8 & Build calculus related utiltites\\
\hline
March 15 & Build logic related utilities\\
\hline
March 22 & Build any extra mathematical utilties\\
\hline
March 29 & Testing\\
\hline
April 5 & Packaging for Windows, OS X, and Linux\\
\hline
April 12 & Extra space to be used as needed\\
\hline
April 19 & Extra space to be used as needed\\
\hline
\end{tabular}
\end{center}

\end{document}
