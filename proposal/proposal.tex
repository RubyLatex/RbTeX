\documentclass{article}

\usepackage{amsmath}
\usepackage{listings}
\usepackage{graphicx}

\newcommand{\inlinecode}[1]{\texttt{#1}}
\newcommand{\luatex}{\inlinecode{lualatex}\ }
\newcommand{\findent}{\leavevmode{\parindent=1.3em\indent}}

\def\RbTeX{{\rm\kern-.125emR\!{\scalebox{0.7}{\lower-.5ex\hbox{B}}}\!T\kern-.1667em\lower.5ex\hbox{E}\kern-.125emX}}

\topmargin=-1in
\textheight=9.2in
\textwidth=168mm
\oddsidemargin=-0.2in
\evensidemargin=-0.2in

\pagenumbering{gobble}
\title{CPSC 498 Proposal: \RbTeX}
\author{Steven Rosendahl}
\date{}
\begin{document}
\maketitle
\section{Abstract}

Modern \LaTeX\ distributions include a tool called \luatex that allows users to dynamically
produce content via use of Lua code. Unfortunately, the Lua standard libraries do not have as much
functionality as other popular scripting languages, such as Ruby. The goal of this project is to
incorporate Ruby into \LaTeX\ in a manner similar to \luatex, but with the power and
simplicity of Ruby over Lua.

\section{Motivation and Description}

\findent The current \luatex specification allows users to use several environments for writing and
running Lua scripts. In addition, \luatex provides a built in library called \inlinecode{tex} that allows
output to be printed straight to the \LaTeX\ document. The library, called \RbTeX, will provide similar
functionality through a gem called \inlinecode{rbtex}. In addition, the entire Ruby standard library will
be available for use; \RbTeX\ documents that need to interact directly with the system will most likely
need to be compiled using the \inlinecode{--shell-escape} flag.\\

To use the library, users will need to have a Ruby version in the path. The code will be pre-processed,
and inserted directly into the \TeX\ code before \inlinecode{pdflatex} is called on the document. In
addition, users will be provided with several ways in which to interact with Ruby from the \TeX\
environment:

\begin{enumerate}
\item \inlinecode{inrbtex\{\}}: This command will provide a way for a use to execute one line of Ruby
code at a time, or call a predefined function.
\item \inlinecode{rbtex\{\}}: This command will provide a way to write multiple lines of Ruby code inside
the \LaTeX\ document. Any functions defined in this section will be globally defined, so they can be
called in the \inlinecode{inrbtex\{\}} environment and in other \inlinecode{rbtex\{\}} environments. This
code will be pulled verbatim from the \LaTeX\ file during the pre-processing stage.
\item \inlinecode{frbtex\{\}}: It may sometimes be convenient for a user to write an external Ruby file
and call it from the \LaTeX\ file, rather than writing the code. The \inlinecode{frbtex\{\}} macro will
allow an external script to be loaded into the document. The pre-processor will copy the provided Ruby
file into the \LaTeX\ document, and will assume that all modules, classes, functions, and variables are
globally defined.
\end{enumerate}

The library will come with a program called \inlinecode{rbtex} that complies the provided \LaTeX\
document, much like the \inlinecode{luatex} command.\\

The program will work in four steps. It will first pre-process the \TeX\ file, scanning for the
appropriate environments. The ruby code will be ripped out and stored in an \inlinecode{.aux} file.
The next step will order the code in the \inlinecode{.aux} file, and attempt to produce a
\inlinecode{.rb} file from the provided \LaTeX\ document. In the third step, the code in the
\inlinecode{.rb} file will be run using the \inlinecode{ruby} command specified in the user's
\inlinecode{\$PATH} variable. Finally, the post-processor will capture any output specified by the
\inlinecode{tex} module in the \inlinecode{rbtex} gem, and place it into the \LaTeX\ document. From
there, \inlinecode{pdflatex} will take over.\\

The pre-processor and post-processor will be written in C++ to allow for quick speed when parsing
out the Ruby code. Shell script (UNIX) and Batch files (Win) will be provided for calling the
program. Standard \inlinecode{pdflatex} flags will be available (they will be simply passed to the
\inlinecode{pdflatex} command at the appropriate time). The final version of the program will be
accessible through the shell command \inlinecode{rblatex tex\_file.tex --shell-escape} and the Windows
equivalent.

\section{Tentative Schedule}
\begin{center}
\begin{tabular}{| c | l |}
\hline
January 19 & Proposal and initial git setup\\
\hline
January 26 & Research luatex implementation\\
\hline
February 2 & Begin writing pre/post-processor\\
\hline
February 9 & Continue writing pre/post-processor\\
\hline
February 16 & Continue writing pre/post-processor\\
\hline
February 23 & Test pre/post-processor\\
\hline
March 1 & Test pre/post-processor\\
\hline
March 8 & Build \TeX\ bindings\\
\hline
March 15 & Build \TeX\ bindings\\
\hline
March 22 & Test entire package\\
\hline
March 29 & Build C program to include in \TeX\ package\\
\hline
April 5 & Clean and final test code\\
\hline
April 12 & Extra space to be used as needed\\
\hline
April 19 & Extra space to be used as needed\\
\hline
\end{tabular}
\end{center}

\end{document}
